%!TEX program = xelatex
\documentclass{article}

% HW information
\title{MTH 9821 Numerical Methods for Finance -- Homework 2}
\author{Bledar Kulemani, Avinash Oza, Bingcheng Wang, Xinlu Xiao, Wenli Dong}
\date{\today}
\makeatletter
\let\newtitle\@title
\let\newauthor\@author
\let\newdate\@date
\makeatother

% insert code
\usepackage{listings}
\usepackage{fontspec}
\newfontfamily\monaco{Monaco}
\usepackage{xcolor}
\definecolor{listinggray}{gray}{0.9}
\definecolor{lbcolor}{rgb}{0.9,0.9,0.9}
\lstset{
backgroundcolor=\color{lbcolor},
    tabsize=4,    
%   rulecolor=,
    language=[GNU]C++,
        basicstyle=\scriptsize,
        upquote=true,
        aboveskip={1.5\baselineskip},
        columns=fixed,
        showstringspaces=false,
        extendedchars=false,
        breaklines=true,
        prebreak = \raisebox{0ex}[0ex][0ex]{\ensuremath{\hookleftarrow}},
        frame=single,
        numbers=left,
        showtabs=false,
        showspaces=false,
        showstringspaces=false,
        identifierstyle=\ttfamily,
        keywordstyle=\color[rgb]{0,0,1},
        commentstyle=\color[rgb]{0.026,0.112,0.095},
        stringstyle=\color[rgb]{0.627,0.126,0.941},
        numberstyle=\color[rgb]{0.205, 0.142, 0.73},
%       \lstdefinestyle{C++}{language=C++,style=numbers}’.
}
\lstset{
    backgroundcolor=\color{lbcolor},
    tabsize=4,
  language=C++,
  captionpos=b,
  tabsize=3,
  frame=lines,
  numbers=left,
  numberstyle=\tiny,
  numbersep=5pt,
  breaklines=true,
  showstringspaces=false,
  basicstyle=\footnotesize,
%  identifierstyle=\color{magenta},
  keywordstyle=\color[rgb]{0,0,1},
  commentstyle=\color{Darkgreen},
  stringstyle=\color{red}
  }

% set size and margins
\usepackage[a4paper, top=2cm, bottom=2cm, left=2cm, right=2cm]{geometry}
\usepackage{amsmath, bm, amssymb} % math
	\DeclareMathOperator{\E}{\mathbb{E}}
	\DeclareMathOperator{\1}{\mathit{1}}
\usepackage{fontspec} % font
	\newfontfamily\menlo{Menlo}
\usepackage[colorlinks, linkcolor=red, anchorcolor=blue, citecolor=red]{hyperref} % hyperlink
\usepackage{fancyhdr} % header and footer
    \pagestyle{fancy}
    \lhead{\newtitle}
    \chead{}
    \rhead{}
    \lfoot{}
    \cfoot{}
    \rfoot{\thepage}
    \renewcommand{\headrulewidth}{0.4pt}
    \renewcommand{\footrulewidth}{0.4pt}
\usepackage{enumitem} % enumerate package
\usepackage[framemethod=TikZ]{mdframed} % frame text
\usepackage{amsthm}
    % R Language Output
    \newcounter{Rop}[section]\setcounter{Rop}{0}
    \renewcommand{\theRop}{\arabic{Rop}}
    \newenvironment{Rop}[2][]{%
    \refstepcounter{Rop}%
    \ifstrempty{#1}%
    {\mdfsetup{%
    frametitle={%
    \tikz[baseline=(current bounding box.east),outer sep=0pt]
    \node[anchor=east,rectangle,fill=blue!20]
    {\strut R~Output~\theRop};}}
    }%
    {\mdfsetup{%
    frametitle={%
    \tikz[baseline=(current bounding box.east),outer sep=0pt]
    \node[anchor=east,rectangle,fill=blue!20]
    {\strut R~Output~\theRop:~#1};}}%
    }%
    \mdfsetup{innertopmargin=10pt,linecolor=blue!20,%
    linewidth=2pt,topline=true,%
    frametitleaboveskip=\dimexpr-\ht\strutbox\relax
    }
    \begin{mdframed}[font=\menlo]\relax%
    \label{#2}}{\end{mdframed}}
\pdfstringdefDisableCommands{% necessary when there are bm and hyperref
    \renewcommand*{\bm}[1]{#1}%
}

% \renewcommand{\thefigure}{\thesection.\arabic{n}} % figure number
\numberwithin{figure}{section} %figure number
\numberwithin{table}{section}

%\setlength{\parindent}{0pt} %indent
%%%%%%%%%%%%%%%%%%%%%%%%%%%%%%%%%%%%%%%%%%%%%%%%%%%%%%%%%%%%
\begin{document}
\maketitle

%%%%%%%%%%%%%%%%%%%%%%%%55Bledar's problems%%%%%%%%%%%%%%%%%%%
\section{Problem 6.7} 
Let \(B_N\) be the N x N  tridiagonal symmetric positive matrix be given by (1). 

% Creating the matrix Bn
\begin{equation}
	\mathbf{B_N} = \left(
	\begin{array}{cccc}
	2 & -1 & \ldots & 0\\
	-1 & \ddots & \ddots & \vdots \\
	\vdots & \ddots & \ddots & -1 \\
	0 & \ldots & -1 & 2
	\end{array} \right)
\end{equation}
and let \(U_N\) be the Cholesky factor  of the matrix \(B_N\) be given by:
\begin{equation}
	U_N(i,i) = \sqrt{\frac{i+1}{i}}
\end{equation}

\begin{equation}
	U_N(i,i+1) = -\sqrt{\frac{i}{i+1}}
\end{equation}

(\(i)\) Show that the solution to a linear system \(B_N*x\) = b, where b and x are the N x 1 column vectors can be obtained by using the explicit pseudocode below:\\\\
	\indent \hspace{5 cm} Function Call:\\
	\indent \hspace{5 cm} x = linear\_solve\_cholesky\_B\_N(b)\\\\
	\indent \hspace{5 cm} Input:\\
	\indent \hspace{5 cm} b = N x 1 column vector \\\\
	\indent \hspace{5 cm} Output:\\
	\indent \hspace{5 cm} x = solution to \(B_N*x\) = b\\\\
	\indent \hspace{5 cm} y(1)= \(\frac{b(1)}{\sqrt{2}}\) \\
	\indent \hspace{5 cm} for \(i\)=2:\(N\) \\
	\indent \hspace{5 cm} \indent y(\(i\)) = \(\frac{b(i)-y(i-1)*\sqrt{\frac{i-1}{i}}}{\sqrt{\frac{i+1}{i}}}\) \\
	\indent \hspace{5 cm} end \\\\
	\indent \hspace{5 cm} x(\(N\)) = \(\frac{y(N) * \sqrt{N}}{\sqrt{N+1}}\)\\
	\indent \hspace{5 cm} for \(i\)=2:\(N\) \\
	\indent \hspace{5 cm} \indent y(\(i\)) = \(\frac{y(i)+x(i+1)*\sqrt{\frac{i}{i+1}}}{\sqrt{\frac{i+1}{i}}}\) \\
	\indent \hspace{5 cm} end \\\\
	
(\(ii)\) What is the operation count for the pseudocode above, and how does it compare to \(8n\)+\(O(1)\), the operation count for the optimal linear solver for the tridiagonal symmetric definite matrices? \\\\

\underline{\textbf {Proof}} \\
\((i)\)
\par We know that \(U_N^T  U_N  x = b\). If we let \(U_N x = y\), then we have \(U_N^T y = b\). Note that \(U_N^T\) is a lower triangular matrix. Therefore, this can be solved using the forward substitution(L,b) algorithm. Its implementation is as follows:\\
\indent \(y(1) = \frac{b(1)}{U_N^T(1,1)} = \frac{b(1)}{\sqrt{B_N(1,1)}} = \frac{b(1)}{\sqrt{2}}\) \\\\
Then, equation (2) becomes:\\
\begin{equation}
	U_N(i,i) = \sqrt{\frac{i+1}{i}} \hspace{1 cm} for \hspace{2 mm} every \hspace{2 mm} i=2:N
\end{equation}

and equation (3) becomes: \\
\begin{equation}
	U_N(i,i+1) = -\sqrt{\frac{i-1}{i}} \hspace{1 cm} for \hspace{2 mm} every \hspace{2 mm} i=2:N
\end{equation}

Note that to move from equation (3) to equation (5) we just let \(i=i-1\).
We then use the forward\_subst\_bidiag(L,b) to find the vector y.\\\\
\indent y(1)= \(\frac{b(1)}{\sqrt{2}}\) \\\\
\indent for \(i\)=2:\(N\) \\
\indent \hspace{1 cm} \(y(i)= \frac{b(i)-y(i-1)L(i,i-1)}{L(i,i)} = \frac{b(i)-y(i-1)U(i-1,i)}{L(i,i)} = \frac{b(i)+y(i-1) \sqrt{\frac{i-1}{i}}}{\sqrt{\frac{i+i}{i}}}\) \indent Note that \(L(i,j)=U(j,i)\)\\ 
\indent end \\
\par Now that we have the vector \(y\), we can use it to find vector \(x\) in \(U_Bx=y\). To solve for x, we use the backward substitution algorithm for bidiagonal matrices. The algorithm is shown in Table 2.4 of the book.\\\\
\indent \( x(N)=\frac{y(N)}{U_N(N,N)} = \frac{y(N)}{\sqrt{\frac{N+1}{N}}} = \frac{y(N)\sqrt{N}}{\sqrt{N+1}} \)\\
\indent for \(i\)=2:\(N\) \\
\indent \hspace{1 cm} \(y(i) = \frac{y(i)-x(i+1)U(i,i+1)}{U(i,i)} = \frac{y(i)-x(i+1)*(-\sqrt{\frac{i}{i+1}})}{\sqrt{\frac{i+1}{i}}} = \frac{y(i)+x(i+1)*\sqrt{\frac{i}{i+1}}}{\sqrt{\frac{i+1}{i}}}\) \\
\indent end \\
If we put these two loops together, then we see that the explicit pseudocode solves the system \(B_Nx=y\).\\\\
\((ii)\) \\
The loop that runs from 2:N has:\\
\indent a) \((N-2)+1\) iterations\\ 
\indent b) 9 operations per iteration.\\\\
The loop that runs from N-1:1 has:\\
\indent a) \((N-1)-1+1\) iterations\\
\indent b) 9 operations per iterations\\\\
Additionally, there are 2 operations needed to calculate \(y(1)\) and 5 operations to calculate \(x(N)\). Therefore, the total no. of ops. is \(18(N-1) + 7 = 18n +O(1)\).\\

\newpage{}
\section {Problem 6.8} 
\par Let \(B_N\) be the \(NxN\) tridiagonal symmetric positive definite matrix given in problem 7. Show that the L and U factors of the matrix  \(B_N\) are the lower triangular bidiagonal matrix and the upper triangular bidiagonal matrix given by:\\
\indent \(L(i,i) = 1\) for every \(i=1:N\) and \(L(i+1,i) = -\frac{i}{i+1}\) for every \(i=1:(N-1)\)\\ 
\indent\(U(i,i) = \frac{i+1}{i}\) for every \(i=1:N\) and \(U(i,i+1) = -1\) for every \(i=1:(N-1)\)\\\\
\underline{\textbf {Proof}} 
\par It is important to notice that the leading principal minors of the matrix \(B_N\) are non-zero. Therefore, we can apply the lu\_no\_pivoting\_tridiag(A) algorithm provided in the Table 2.7. We first initialize the lower triangular matrix L to Identity matrix and the upper triangular matrix U to the Zero matrix, i.e.\\\\
\indent \hspace{5 cm} \(L=\textbf{I}_{NxN}\)\\
\indent \hspace{5 cm} \(U=\textbf{0}_{NxN}\)\\\\
\indent \hspace{5 cm}for \(i=1:(N-1)\)\\
\indent \hspace{5 cm} \indent \(L(i,i) = 1\)\\
\indent \hspace{5 cm} \indent \(L(i+1,i) = \frac{A(i+1,i)}{A(i,i)}\)\\
\indent \hspace{5 cm} \indent \(U(i,i) = A(i,i)\)\\
\indent \hspace{5 cm} \indent \(U(i+1,i+1) = A(i,i+1)\)\\
\indent \hspace{5 cm} \indent \(A(i+1,i+1) = A(i+1,i+1)-L(i+1,i)*U(i,i+1)\)\\
\indent \hspace{5 cm} end\\
\indent \hspace{5 cm} \(L(n,n) = 1; \indent U(n,n) = A(n,n)\)\\\\
\indent To prove the claim, we will use induction.\\\\
Basis: \(i=1\)\\
Using the algorithm above, we have:\\\\
\indent \(L(1,1)=1\)\\
\indent \(L(2,1)= \frac{A(2,1)}{A(1,1)}= \frac{-1}{2}\) and \\\\
\indent \(U(1,1)= A(1,1) = 2\)\\
\indent \(U(1,2)=A(1,2)=-1\)\\\\
As we can see from the basis step, the case when \(i=1\) is true. Let's assume that the claim is true for \(i=k\), i.e.\\
\indent \(L(k,k)=1\)\\
\indent \(L(k+1,k)= \frac{-k}{k+1}\) and \\\\
\indent \(U(k,k)= \frac{k+1}{k}\)\\
\indent \(U(k,k+1)=-1\)\\\\
Then, we need to check whether the claim is true for \(i=k+1\), i.e.\\ 
\indent \(L(k+1,k+1)=1\)\\
\indent \(L(k+2,k+1)= \frac{-(k+1)}{k+2}\) and \\\\
\indent \(U(k+1,k+1)= \frac{k+2}{k+1}\)\\
\indent \(U(k+1,k+2)=-1\)\\\\
After step k, the LU decomposition without row pivoting updates the block of A that is on the bottom right corner of element with index (k,k), i.e. \\\\
\indent \(A(k+1:n,k+1:n) = A(k+1:n,k+1:n) - L(k+1:n,k)*U(k,k+1:n) \)\\\\
Particularly,\\
\indent \(A(k+1,k+1)= A(k+1,k+1) - L(k+1,k)*U(k,k+1)\)\\
\indent \(A(k+1,k+1)= 2 - \frac{-k}{k+1} *(-1)\) \indent\indent (by induction hypothesis)\\
\indent \(A(k+1,k+1)= \frac{k+2}{k+1} \)\\\\
Additionally, \\
\indent \(A(k+1,k+2)= A(k+1,k+2) - L(k+2,k)*U(k,k+2)\)\\
\indent \(A(k+1,k+1)= -1 - \frac{A(k+2,k)}{U(k,k)}*U(k,k+2) \) \indent (by induction hypothesis)\\
\indent \(A(k+1,k+1)= -1 \) \indent  (since \(A(k+2,k)=0)\)\\\\
Therefore we have:\\
\indent \(U(k+1,k+1) = A(k+1,k+1) = \frac{k+2}{k+1}\)\\
\indent \(U(k+1,k+2)=A(k+1,k+2) = -1\) \indent and \\\\
\indent \(L(k+2,k+1) = \frac{A(k+2,k+1)}{A(k+1,k+1)}\)\\
\indent \(L(k+2,k+1) = \frac{A(k+1,k+2)}{A(k+1,k+1)}\) \indent (by symmetry)\\
\indent \(L(k+2,k+1) = \frac{-1}{\frac{k+2}{k+1}} = -\frac{k+1}{k+2}\)\\
\indent \(L(k+1,k+1)=1 \)\indent (since L is a LU factor)\\\\
Then, by induction, the proof follows. Note that the operation count is \(3n + O(1).\)

\newpage{}
\section{Problem 6.9}
\par Let \(B_N\) be the \(NxN\) tridiagonal symmetric positive definite matrix given in problem 7, and let L and U be the LU factors of the matrix  \(B_N\) given in problem 8.\\
\((i)\) Show that the solution to a linear system \(B_N*x=b\) where \(b\) and \(x\) are the \(Nx1\) column vectors can be obtained by using the pseudocode below:\\\\
			\indent \hspace{5 cm}	Function Call:\\
			\indent \hspace{5 cm}	x = linear\_solve\_LU\_B\_N(b)\\\\
			\indent \hspace{5 cm}	Input:\\
			\indent \hspace{5 cm}	b = N x 1 column vector \\\\
			\indent \hspace{5 cm}	Output:\\
			\indent \hspace{5 cm}	x = solution to \(B_N*x\) = b\\\\
			\indent \hspace{5 cm}	y(1)= \(b(1)\) \\
			\indent \hspace{5 cm}	for \(i\)=2:\(N\) \\
			\indent \hspace{5 cm}	\indent \(y(i) = b(i) + \frac{(i-1)*y(i-1)}{i}\)\\ 
			\indent \hspace{5 cm}	end \\\\
			\indent \hspace{5 cm}	\(x(N) = \frac{N_y(N)}{N+1}\)\\
			\indent \hspace{5 cm}	for \(i\)=2:\(N\) \\
			\indent \hspace{5 cm} \indent y(\(i\)) = \(\frac{i*(y(i)+x(i+1))}{i+1}\) \\
			\indent \hspace{5 cm} end \\\\
\underline{\textbf {Proof}} \\
	\indent \(L*U*x=b\)\\
	\indent \(L*Y=b\)\\\\
We will first use the forward substitution to find the vector b. \\\\
\(y(1) = \frac{b(1)}{L(1,1)}\)\\
for \(i=2:N\)\\
	\indent \(y(i) = \frac{b(i)-L(i,i-1)*y(i-1)}{L(i,i)}\)\\
	\indent \(y(i) = \frac{b(i)-(-\frac{i-1}{i})*y(i-1)}{1}\)\\
	\indent \(y(i) = b(i)+\frac{i-1}{i}*y(i-1)\)\\
end\\\\
Then, we can use backward\_subst\_bidiag to solve for the vector x.\\
\indent \(x(N) = \frac{y(N)}{U(N,N)} = \frac{y(N)}{\frac{N+1}{N}}= \frac{y(N)*N}{N+1}\\\)\\
for \(i =(n-1):1\)\\
\indent \(x(i)=\frac{y(i) - U(i,i+1)*x(i+1)}{U(i,i)} = \frac{y(i) + x(i+1)}{\frac{i+1}{i}} = \frac{i*(y(i) + x(i+1))}{i+1}   \)\\
end\\\\
After putting these two pieces of code together, we see that the algorithm is produced.
Operation count:\\
	\indent First for loop: \(4*(N+1)\)\\
	\indent Second for loop: \(4*(N+1)\)\\
	\indent Operation count in calculating \(x(N)\) is 3\\
	\indent Operation count in calculating \(y(1)\) is 0, since \(y(1)=b(1)\)\\ 
The total operation count for this algorithm is: \(8*(N-1) + 3 = 8N - 5 = 8n+O(1)\)

\newpage{}
\section{Problem 6.10}

\par Write an explicit optional pseudocode for solving linear systems corresponding to the same tridiagonal symmetric positive definite matrix. In other words, write a pseudocode for solving p linear systems \(A*x_i = b_i\), for \(i=1:p\) where \(A\) is tridiagonal symmetric positive definite matrix.\\\\
\underline{\textbf {Solution}} \\

\par An optimal explicit code to solve p linear systems corresponding to the same tridiagonal symmetric positive definite matrix is outlined below:\\
	\indent - Find the LU decomposition of the matrix\\
	\indent - Use a "for" loop to solve the system using forward and backward substitutions for each vector \(b_i\)\\\\ 

Input:\\
A - tridiagonal symmetric positive definite matrix of size n\\
\(b_i\) - column vector of size n\\\\

Output:\\
\(x_i\) - solution corresponding to the vector \(b_i\)\\
for \(i=1:n-1\)\\
	\indent \(L(i,i) = 1;  \indent L(i+1,i) = \frac{A(i+1,i)}{A(i,i)}\)\\
	\indent \(U(i,i) = A(i,i);  \indent U(i,i+1) = A(i,i+1)\)\\
	\indent \(A(i+1,i+1) = A(i+1,i+1)- L(i+1,i)*U(i,i+1) \)\\
end\\

\(L(n,n)=1\) \indent \(U(n,n)=A(n,n)\)\\
// At this point we have the LU decomposition, and it will be used by the solvers\\
for \(i=1:p\)\\
	\indent \(y(1)=b_i(1)\)\\
	\indent for \(j=2:n\)\\
	\indent \hspace{1 cm} \(y(j)=b_i(j)-L(j,j-1)*y(j-1)\) // forward substitution for \(L*y=b_i\)\\
end\\

	\indent \(x_i(n)=\frac{y(n)}{U(n,n)}\)\\
	\indent \hspace{1 cm}
for \(j=(n-1):i\)\\
	\indent \hspace{1 cm} \(x_i(j)=\frac{y(j)-U(j,j+1)*x_i(j+1)}{U(j,j)}\)\\
	\indent end	\indent // backward substitution for \(U*x_i=y\)\\
end\\\\
Now we calculate the operation count:\\
\indent For the LU decomposition, there are \(3*(n-1)\). For the forward substitution, there are \(2*n-2\). For the backward substitution, there are \(3*n-2\). The number of operations for the forward and backward substitution is multiplied by p, since the code runs p-times.\\ 
Therefore, the total number of the ops is: \(p*(3n-2+2n-2)+3n-3\) = \(5*n*p +3*n -4*p -3\)\\
%%%%%%%%%%%%%%%%%%%%%%%%%%%%%%%%%%%%%%%%%%%%%%%%%%%%%%%%%%%%%%
%%%%%%%%%%%%%%%%%%%%%%%%%%%%
\newpage
\section{Ch 8 \#3}
\subsection{Subquestion i}

The mid prices of the options are shown in Table \ref{tab:midPrice}.

	\begin{table}[hbtp]
	\centering
    \caption{\label{tab:midPrice}%
    Mid Prices of Options}
    \small
    \begin{tabular}{ccc}
    \hline\hline
    \bf{Strike} & \bf{Mid Prices Call} & \bf{Mid Prices Put}\\ 
    \hline
1450 & 432.8 & 9.55\\
1500 & 385.75 & 12.35\\
1550 & 339.55 & 16.05\\
1600 & 294.5 & 20.85\\
1675 & 229.7 & 30.55\\
1700 & 208.8 & 34.55\\
1750 & 168.8 & 44.45\\
1775 & 150.1 & 50.45\\
1800 & 131.7 & 57.3\\
1825 & 114.25 & 64.65\\
1850 & 97.7 & 73.05\\
1875 & 82.35 & 82.45\\
1900 & 68.05 & 93.3\\
1925 & 55.05 & 105.05\\
1975 & 33.65 & 133.55\\
2000 & 25.2 & 150.55\\
2050 & 13 & 188.3\\
2100 & 6.05 & 231.2\\
    \hline\hline
    \end{tabular}
    \end{table}
    
The OLS formula is
	\begin{equation*}
	y = PVF - disc \times K
	\end{equation*}
where $y = C - P$, $C$ is the mid prices of the call options, $P$ is the mid prices of the put options, and $K$ is the strike price. By using OLS, we get
	\begin{equation*}
	y = 1869.4031 - 0.9972K
	\end{equation*}
	
Thus $PVF = 1869.4031$, $disc = 0.9972$.

\subsection{Subquestion ii}

The original Black-Scholes Formula is
	\begin{align*}
	C_{BS}(S, K, T, \sigma, r, q) &= Se^{-qT}N(d_1) - Ke^{-rT}N(d_2) \\
	P_{BS}(S, K, T, \sigma, r, q) &= -Se^{-qT}N(-d_1) + Ke^{-rT}N(-d_2)
	\end{align*}
where $S$ is the spot price of the index corresponding to these options, $q$ is the dividend rate, $r$ is the risk-free rate. The function $N(\cdot)$ is the standard normal cdf, and
	\begin{align*}
	d_1 &= \frac{\log (S/K) + (r-q+\sigma^2/2)T}{\sigma\sqrt{T}} \\
	d_2 &= \frac{\log (S/K) + (r-q-\sigma^2/2)T}{\sigma\sqrt{T}}
	\end{align*}

We can rewrite Black-Scholes formula using $PVF$ and $disc$, then
	\begin{align*}
	C_{BS} &= PVF\cdot N(d_1) - K\cdot disc\cdot N(d_2) \\
	P_{BS} &= -PVF\cdot N(-d_1) + K\cdot disc\cdot N(-d_2) 
	\end{align*}
and
	\begin{align*}
	d_1 & =  \frac{\log \left( \frac{PVF}{K\cdot disc}\right)}{\sigma\sqrt{T}} + \frac{\sigma\sqrt{T}}{2} \\
	d_2 & =  \frac{\log \left( \frac{PVF}{K\cdot disc}\right)}{\sigma\sqrt{T}} - \frac{\sigma\sqrt{T}}{2} 
	\end{align*}
	
Let $C_{BS}$ and $P_{BS}$ be the mid prices of the call and put options. We can use Newton's method to calculate $\sigma$, which will be the implied volatility. The implied volatility is shown in Table \ref{tab:impVol}.

	\begin{table}[hbtp]
	\centering
    \caption{\label{tab:impVol}%
    Implied Volatility}
    \small
    \begin{tabular}{ccc}
    \hline\hline
    $\mathbf{K}$ & $\mathbf{\sigma^2_{imp,~c}}$ & $\mathbf{\sigma^2_{imp,~p}}$\\ 
    \hline
1450 & 0.0473548237418695 & 0.047763472746148\\
1500 & 0.0431902618976051 & 0.043488616065135\\
1550 & 0.039308491461635 & 0.0395996294773441\\
1600 & 0.0357431718481531 & 0.0359616052698727\\
1675 & 0.0307945166866716 & 0.0307075462798457\\
1700 & 0.0290614122800266 & 0.0289593032638681\\
1750 & 0.0258423104463723 & 0.0257818527826046\\
1775 & 0.0244890865947133 & 0.0242987325201529\\
1800 & 0.0229048005647355 & 0.0229011137326154\\
1825 & 0.0214286123535537 & 0.0213593275657909\\
1850 & 0.0199790343583373 & 0.0199229868497235\\
1875 & 0.0186480463469214 & 0.018513108797099\\
1900 & 0.0173158231345902 & 0.0172811093183057\\
1925 & 0.016049424826671 & 0.015938192074204\\
1975 & 0.0138591431080441 & 0.0137615241485985\\
2000 & 0.0128711121494107 & 0.0130099516032848\\
2050 & 0.0112692419430679 & 0.0114918512531617\\
2100 & 0.0101438878456743 & 0.0104521034753514\\
    \hline\hline
    \end{tabular}
    \end{table}

From Table \ref{tab:impVol} we can see that under the same strike price, the implied volatilities of calls and puts are very close.
\section{Ch 8 \#7}
\subsection{Subquestion i}

By using OLS regression, we get
	\begin{equation*}
	T_{3,LR} = 0.0123 + 0.1272T_2 + 0.3340T_5 + 0.5298T_{10}
	\end{equation*}

The approximation error
	\begin{equation*}
	\mathrm{error}_{LR} = ||T_3 - T_{3,LR} || = 0.04301298
	\end{equation*}

\subsection{Subquestion ii}
By using linear interpolation, we get $T_{3, linear_interp} = $ [4.650000, 4.770000, 4.773333, 4.783333, 4.783333, 4.796667, 4.780000, 4.786667, 4.810000
4.790000, 4.803333, 4.800000, 4.783333, 4.760000, 4.763333].

Thus the approximation error
	\begin{equation*}
	\mathrm{error}_{linear_interp} = ||T_3 - T_{3,linear_interp} || = 0.2066129
	\end{equation*}
	
\subsection{Subquestion iii}
By using cubic spline interpolation, we get $T_{3, cubic_interp} = $ [4.641333, 4.762000, 4.765889, 4.780889, 4.781222, 4.789111, 4.773667, 4.781778, 4.805667, 4.786000, 
4.799556, 4.795333, 4.779556, 4.756333, 4.758222].

Thus the approximation error
	\begin{equation*}
	\mathrm{error}_{cubic_interp} = ||T_3 - T_{3,cubic_interp} || = 0.1864353
	\end{equation*}

\subsection{Subquestion iv}
The OLS regression has the least approximation error, and the linear interpolation has the approximation error. That is because OLS estimator is BLUE. It has very little residual, while linear interpolation is very rough.

\section{Ch 8 \#8}
\subsection{Subquestion i}

The weekly percentage returns of these stocks are listed in Table \ref{tab:weekly_ret}.

	\begin{table}[hbtp]
	\centering
    \caption{\label{tab:weekly_ret}%
    Weekly returns of some stocks}
    \small
    \begin{tabular}{cccccccccc}
    \hline\hline
\bf{date} & \bf{JPM} & \bf{GS} & \bf{MS} & \bf{BAC} & \bf{RBS} & \bf{CS} & \bf{UBS} & \bf{RY} & \bf{BCS}\\
    \hline
2012-01-23 & 0.0359 & 0.13 & 0.119 & 0.103 & 0.183 & 0.183 & 0.178 & 0.0391 & 0.139\\
2012-01-30 & 0.0289 & 0.0515 & 0.0944 & 0.0758 & 0.0436 & 0.0441 & 0.0438 & 0.0248 & 0.0718\\
2012-02-06 & -0.0176 & -0.029 & -0.0317 & 0.0294 & -0.0352 & -0.0768 & -0.0523 & -0.00343 & -0.019\\
2012-02-13 & 0.0231 & 0.0157 & -0.0256 & -0.00622 & 0.00912 & 0.0422 & 0.0305 & -0.00134 & 0.0704\\
2012-02-20 & 0 & 0 & 0 & 0 & 0 & 0 & 0 & 0 & 0\\
2012-02-27 & 0.0561 & 0.0381 & -0.0152 & 0.015 & -0.00226 & 0.017 & -0.0324 & 0.0689 & 0.0316\\
2012-03-05 & 0.00981 & -0.0223 & -0.0261 & -0.00986 & -0.0668 & -0.0303 & -0.0284 & 0.00501 & -0.0625\\
2012-03-12 & 0.0862 & 0.0481 & 0.063 & 0.217 & 0.091 & 0.116 & 0.0719 & 0.0214 & 0.0667\\
2012-03-19 & 0.0133 & 0.0264 & 0.0407 & 0.00512 & -0.00222 & -0.014 & -0.014 & -0.00872 & -0.0212\\
2012-03-26 & 0.0181 & -0.0143 & -0.0337 & -0.0275 & -0.0145 & -0.0238 & -0.0149 & -0.000352 & -0.0421\\
2012-04-02 & -0.0294 & -0.0513 & -0.0636 & -0.0356 & -0.0939 & -0.0655 & -0.0655 & -0.0137 & -0.0827\\
2012-04-09 & -0.0254 & -0.0246 & -0.0608 & -0.0597 & -0.0187 & -0.0342 & -0.0477 & -0.0212 & -0.0218\\
2012-04-16 & -0.0113 & -0.023 & 0.0117 & -0.037 & -0.0267 & 0.0133 & 0.00162 & 0.0283 & 0.00446\\
2012-04-23 & 0.0145 & 0.0175 & -0.0276 & -0.0132 & 0.0392 & -0.0549 & 0.0161 & 0.0186 & 0.0636\\
2012-04-30 & -0.0368 & -0.0474 & -0.0557 & -0.062 & -0.00881 & -0.0883 & -0.0214 & -0.0506 & -0.0709\\
2012-05-07 & -0.115 & -0.063 & -0.0659 & -0.0246 & -0.0698 & -0.0277 & -0.0114 & -0.0229 & -0.0404\\
2012-05-14 & -0.094 & -0.065 & -0.107 & -0.0704 & -0.145 & -0.0707 & -0.0755 & -0.0606 & -0.134\\
2012-05-21 & 0.000303 & 0.0127 & -0.00752 & 0.0186 & 0.0399 & 0.00715 & 0.0266 & -0.0356 & 0.0243\\
2012-05-28 & 0 & 0 & 0 & 0 & 0 & 0 & 0 & 0 & 0\\
2012-06-04 & 0.00546 & -0.0176 & 0.0348 & 0.0589 & 0.0706 & 0.0264 & 0.013 & 0.00352 & 0.0431\\
2012-06-11 & 0.0401 & 0.0118 & 0.0425 & 0.045 & 0.122 & -0.0697 & 0.0239 & 0.0243 & 0.0649\\
2012-06-18 & 0.0273 & -0.0212 & -0.0105 & 0.00507 & -0.0268 & -0.00425 & -0.00334 & 0.0101 & -0.00713\\
2012-06-25 & -0.00734 & 0.0238 & 0.0312 & 0.0303 & -0.108 & -0.0245 & -0.0201 & 0.0102 & -0.183\\
2012-07-02 & -0.0432 & -0.00409 & -0.0303 & -0.0636 & -0.0779 & -0.0262 & -0.0589 & 0.0124 & -0.00293\\
2012-07-09 & 0.0642 & 0.0205 & -0.00639 & 0.0209 & 0.0319 & -0.0213 & -0.0299 & -0.00234 & -0.00294\\
2012-07-16 & -0.0603 & -0.0335 & -0.0907 & -0.096 & -0.0124 & -0.0287 & -0.0505 & 0 & -0.0324\\
2012-07-23 & 0.0883 & 0.0795 & 0.0597 & 0.034 & 0.0814 & 0.0543 & 0.0798 & 0.00665 & 0.066\\
2012-07-30 & -0.0216 & -0.00652 & 0.0215 & 0.0164 & -0.0145 & -0.0364 & -0.0173 & -0.000194 & 0.0105\\
2012-08-06 & 0.0243 & 0.0202 & 0.0602 & 0.0418 & 0.0338 & 0.0221 & 0.0121 & 0.00369 & 0.0858\\
2012-08-13 & 0.000272 & 0.00565 & -0.00137 & 0.0336 & 0.0355 & 0.0381 & 0.0128 & 0.0557 & 0.0477\\
2012-08-20 & 0.00518 & 0.0131 & -0.00206 & 0.02 & -0.0274 & 0.0526 & 0.0145 & -0.00623 & -0.0174\\
2012-08-27 & -0.000813 & 0.0117 & 0.0302 & -0.0209 & 0.0141 & 0.0026 & -0.00446 & 0.033 & -0.0194\\
2012-09-03 & 0 & 0 & 0 & 0 & 0 & 0 & 0 & 0 & 0\\
2012-09-10 & 0.119 & 0.148 & 0.216 & 0.197 & 0.253 & 0.2 & 0.209 & 0.0318 & 0.273\\
2012-09-17 & -0.0167 & -0.0382 & -0.0636 & -0.0461 & -0.0111 & -0.00995 & -0.0415 & -0.00658 & -0.0263\\
2012-09-24 & -0.00986 & -0.026 & -0.0199 & -0.0307 & -0.0662 & -0.0756 & -0.0573 & 0 & -0.0381\\
2012-10-01 & 0.0381 & 0.0495 & 0.0454 & 0.0555 & 0.0168 & 0.0671 & 0.0509 & 0.0223 & 0.0454\\
2012-10-08 & -0.00216 & 0.00746 & -0.0109 & -0.0215 & 0.0213 & -0.0062 & -0.0148 & -0.0123 & 0.0193\\
2012-10-15 & 0.0168 & 0.0285 & 0.0127 & 0.0351 & 0.0347 & 0.0473 & 0.0349 & 0.0162 & 0.0088\\
    \hline\hline
    \end{tabular}
    \end{table}
    
\subsection{Subquestion ii}

By using OLS regression, we get

	\begin{multline*}
	JPM =     -0.003222    +    0.755635~ GS      -0.064294~ MS   +    0.302755~ BAC    +    0.227142~  RBS \\
     -0.094029~ CS     -0.398494~ UBS   +    0.147026~ RY    +   0.010758~ BCS
	\end{multline*}
where those tickers represent the weekly returns of the corresponding tickers.	
	
The approximation error is 0.1319426.

\subsection{Subquestion iii}

By using OLS regression, we get

	\begin{equation*}
	JPM =     -0.001105 +     0.614013~GS    -0.067123~MS+     0.251915~BAC
	\end{equation*}
	where those tickers represent the weekly returns of the corresponding tickers.
	
The approximation error is 0.1514875.

\subsection{Subquestion iv}

	\begin{multline*}
	JPM =     8.83982    +    0.05480~ GS      -0.06069~ MS   +    1.46197~ BAC    +    0.93938~  RBS \\
     0.85709~ CS     -2.24728~ UBS   +    0.28110~ RY    -   0.07965~ BCS
	\end{multline*}
where those tickers represent the prices of the corresponding tickers.	
	
The approximation error is 6.443954. Compared to subquestion ii, the approximation error is much larger.


\end{document}
