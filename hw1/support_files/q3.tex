%!TEX program = xelatex
\documentclass{article}

% HW information
\title{MTH 9821 Numerical Methods for Finance -- Homework 1}
\author{Wenli Dong, Bledar Kulemani, Avinash Oza, Bingcheng Wang, Xinlu Xiao}
\date{\today}
\makeatletter
\let\newtitle\@title
\let\newauthor\@author
\let\newdate\@date
\makeatother

% insert code
\usepackage{listings}
\usepackage{fontspec}
%\newfontfamily\monaco{Monaco}
\usepackage[scaled]{beramono}
\usepackage{xcolor}


% set size and margins
\usepackage[a4paper, top=1in, bottom=1in, left=1in, right=1in]{geometry}
\usepackage{amsmath, bm, amssymb} % math
	\DeclareMathOperator{\E}{\mathbb{E}}
	\DeclareMathOperator{\1}{\mathit{1}}
\usepackage{fontspec} % font
%	\newfontfamily\menlo{Menlo}
	\usepackage[scaled]{beramono}
\usepackage[colorlinks, linkcolor=red, anchorcolor=blue, citecolor=red]{hyperref} % hyperlink
\usepackage{fancyhdr} % header and footer
    \pagestyle{fancy}
    \lhead{\newtitle}
    \chead{}
    \rhead{}
    \lfoot{}
    \cfoot{}
    \rfoot{\thepage}
    \renewcommand{\headrulewidth}{0.4pt}
    \renewcommand{\footrulewidth}{0.4pt}
\usepackage{enumitem} % enumerate package
\usepackage[framemethod=TikZ]{mdframed} % frame text
\usepackage{amsthm}
    % R Language Output
    \newcounter{Rop}[section]\setcounter{Rop}{0}
    \renewcommand{\theRop}{\arabic{Rop}}
    \newenvironment{Rop}[2][]{%
    \refstepcounter{Rop}%
    \ifstrempty{#1}%
    {\mdfsetup{%
    frametitle={%
    \tikz[baseline=(current bounding box.east),outer sep=0pt]
    \node[anchor=east,rectangle,fill=blue!20]
    {\strut R~Output~\theRop};}}
    }%
    {\mdfsetup{%
    frametitle={%
    \tikz[baseline=(current bounding box.east),outer sep=0pt]
    \node[anchor=east,rectangle,fill=blue!20]
    {\strut R~Output~\theRop:~#1};}}%
    }%
    \mdfsetup{innertopmargin=10pt,linecolor=blue!20,%
    linewidth=2pt,topline=true,%
    frametitleaboveskip=\dimexpr-\ht\strutbox\relax
    }
    \begin{mdframed}[font=\menlo]\relax%
    \label{#2}}{\end{mdframed}}
% \renewcommand{\thefigure}{\thesection.\arabic{n}} % figure number
\numberwithin{figure}{section} %figure number
\pdfstringdefDisableCommands{% necessary when there are bm and hyperref
    \renewcommand*{\bm}[1]{#1}%
}

%\setlength{\parindent}{0pt} %indent
%%%%%%%%%%%%%%%%%%%%%%%%%%%%%%%%%%%%%%%%%%%%%%%%%%%%%%%%%%%%
\begin{document}
\maketitle
%%%%%%%%%%%%%%%%%%%%%%%%%%%%
\section{Question 3}
Let 
    \[
        A = 
        \begin{bmatrix}
            2 & -1 & 1 \\
            -2 & 1 & 3 \\
            4 & 0 & -1 \\
        \end{bmatrix}
    \]

\subsection{Subquestion i} 
Show that the 2 x 2 leading principal minor of A is 0, i.e., show that
    \[
        det \begin{bmatrix}
            2 & -1 \\
            -2 & -1 \\
        \end{bmatrix}  = 0
    \]

\textbf{Solution} : The determinant of the second leading principal minor is:
    \[
        det \begin{bmatrix}
            2 & -1 \\
            -2 & -1 \\
        \end{bmatrix}  = 2(1) - (-2)(-1) = 0
    \]

\subsection{Subquestion ii} 
Attempt to do the LU decomposition and show division by U(2,2) cannot be done.

\textbf{Solution}: We apply the LU decomposition and find 
    \[
        U(1,1) = A(1,1) = 2 \\
        U(1,2) = A(1,2) = -1 \\
        U(1,3) = A(1,3) = 1 \\
    \]
We calculate the first column of L as:
    \[
        L(k,1) = \frac{A(k,1)}{U(1,1)}, \forall k = 1:n \\ 
    \]

Then,
    \[
        L(1,1) = \frac{2}{2} = 1, L(2,1) = \frac{-2}{2} = -1, L(3,1) = \frac{4}{2}
    \]
The current LU decomposition is:
    \[
        L = 
        \begin{bmatrix}
            1 & 0 & 0 \\
            -1 & 1 & 0 \\
            2 & L(2,2) & 1 \\
        \end{bmatrix} ,
        U = 
            \begin{bmatrix}
                2 & -1 & 1 \\
                0 & U(2,2) & U(2,3) \\
                0 & 0 & U(3,3) \\
            \end{bmatrix}
    \]
We then update the matrix A by taking out the first column and row of A, L and U. We are left with:
    \[
    \begin{aligned}
        UpdatedA 
        &= 
            \begin{bmatrix}
                1 & 3 \\
                0 & -1 \\
            \end{bmatrix} - 
            \begin{bmatrix}
                -1  \\
                2  \\
            \end{bmatrix}
            \begin{bmatrix}
                -1 & 1 \\
            \end{bmatrix} \\ 
        &= 
            \begin{bmatrix}
                1 & 3 \\
                0 & -1 \\
            \end{bmatrix} - 
            \begin{bmatrix}
                1 & 1 \\
                -2 & 2 \\
            \end{bmatrix} \\
        &= 
            \begin{bmatrix}
                0 & 2 \\
                2 & -3 \\
            \end{bmatrix} 
    \end{aligned}
    \]

We then write:
    \[
        \begin{bmatrix}
            1 & 0 \\
            L(3,2) & 1 \\
        \end{bmatrix}
        \begin{bmatrix}
            U(2,2) & U(2,3) \\
            0 & U(3,3) \\
        \end{bmatrix} =
        \begin{bmatrix}
            0 & 2 \\
            2 & -3 \\
        \end{bmatrix}
    \]
Now,
    \[
        U(2,2) = 0, L(3,2) = \frac{A(3,2)}{U(2,2)}
    \]
which causes the LU decomposition to breakdown with division by zero.



Let $L_1$ and $L_2$ be nonsingular lower triangular matrices and let $U_1$ and $U_2$ be nonsingular upper triangular matrices. If $L_1U_1 = L_2U_2$, show that there exists a nonsingular diagonal matrix D such that 
\begin{equation} \label{eq:1}
L_1 = L_2D \qquad \mathrm{and} \qquad U_1 = D^{-1}U_2
\end{equation}

\section{Question 2}

The sample correlation matrix of the daily percentage returns is
	\[
	\begin{bmatrix}
	1 & 0.879 & 0.829 & 0.598 & 0.912 & 0.944 & 0.921 & 0.697 & 0.518\\
0.879 & 1 & 0.802 & 0.744 & 0.943 & 0.963 & 0.969 & 0.68 & 0.523\\
0.829 & 0.802 & 1 & 0.581 & 0.814 & 0.853 & 0.823 & 0.506 & 0.468\\
0.598 & 0.744 & 0.581 & 1 & 0.711 & 0.717 & 0.728 & 0.507 & 0.432\\
0.912 & 0.943 & 0.814 & 0.711 & 1 & 0.982 & 0.974 & 0.789 & 0.565\\
0.944 & 0.963 & 0.853 & 0.717 & 0.982 & 1 & 0.993 & 0.726 & 0.529\\
0.921 & 0.969 & 0.823 & 0.728 & 0.974 & 0.993 & 1 & 0.713 & 0.514\\
0.697 & 0.68 & 0.506 & 0.507 & 0.789 & 0.726 & 0.713 & 1 & 0.466\\
0.518 & 0.523 & 0.468 & 0.432 & 0.565 & 0.529 & 0.514 & 0.466 & 1
	\end{bmatrix}
	\]
	
\section{Question 3}
\subsection{Subquestion i}
As is shown in \ref{sec:1}, we have already had equation \eqref{eq:1}. And we can see from Table \ref{tab:Parameters_of_Star} and Figure 1 that blablablabla.

	\begin{table}[hbtp]
	\centering
    \caption{\label{tab:Parameters_of_Star}%
    Initial Properties of the External Star.}
    \small
    \begin{tabular}{r|rrrrrrrr}
    \hline\hline
    NO. & $d_e$ (g/$\mathrm{cm}^3$) & $m_e$ ($\mathrm{M}_\odot$) & $x$ (AU) & $y$ (AU) & $z$ (AU) & $v_x$ (km/s) & $v_y$ (km/s) & $v_z$ (km/s) \\ 
    \hline
    1 & 1 & 1 & 100 & 254.558 & 0 & 0 & -2 & 0 \\
    2 & 1 & 1 & 100 & -254.558 & 0 & 0 & 2 & 0 \\
    3 & 1 & 1 & 100 & 254.558 & 0 & 0 & -3 & 0 \\
    4 & 1 & 1 & 100 & -254.558 & 0 & 0 & 3 & 0 \\
    5 & 1 & 1 & 200 & 509.117 & 0 & 0 & -2 & 0 \\
    6 & 1 & 1 & 200 & -509.117 & 0 & 0 & 2 & 0 \\
    7 & 1 & 0.5 & 100 & 254.558 & 0 & 0 & -3 & 0 \\
    8 & 1 & 0.5 & 100 & -254.558 & 0 & 0 & 3 & 0 \\
    \hline\hline
    \end{tabular}
    \end{table}


\subsection{Subquestion ii}
The output is shown in Output 
\subsection{Subquestion iii}

The R code is like

\begin{lstlisting} [language = R]
data = read.csv("MCD_PriceDaily.csv") # import data
adjPrice = data[ ,7] # get the adjusted price of McDonald’s
LogRet = diff(log(adjPrice)) # get the log return of the adjusted price
library(MASS) # import library MASS
library(fGarch) # import library fGarch
# Maximum-likelihood fitting of t-distribution
fit.T = fitdistr(LogRet, "t") 
params.T = fit.T$estimate # get the estimator value
mean.T = params.T[1] # get the mean of t-dist
# get the UNBIASED standard deviation of t-dist
sd.T = params.T[2] * sqrt(params.T[3] / (params.T[3] - 2)) 
# get the degrees of freedom
nu.T = params.T[3]

# plot the histogram of lor return and the fitting desity of t-distribution
x = seq(-0.04, 0.04, by = 0.0001)
hist(LogRet, 80, freq = FALSE)
lines(x, dstd(x, mean = mean.T, sd = sd.T, nu = nu.T),
         lwd = 2, lty = 2, col = "red")
\end{lstlisting}


\end{document}
