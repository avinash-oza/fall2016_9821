%!TEX program = xelatex
\documentclass{article}

% HW information
\title{MTH 9821 Numerical Methods for Finance -- Homework 1}
\author{Wenli Dong, Bledar Kulemani, Avinash Oza, Bingcheng Wang, Xinlu Xiao}
\date{\today}
\makeatletter
\let\newtitle\@title
\let\newauthor\@author
\let\newdate\@date
\makeatother

% insert code
\usepackage{listings}
\usepackage{fontspec}
%\newfontfamily\monaco{Monaco}
\usepackage[scaled]{beramono}
\usepackage{xcolor}


% set size and margins
\usepackage[a4paper, top=1in, bottom=1in, left=1in, right=1in]{geometry}
\usepackage{amsmath, bm, amssymb} % math
	\DeclareMathOperator{\E}{\mathbb{E}}
	\DeclareMathOperator{\1}{\mathit{1}}
\usepackage{fontspec} % font
%	\newfontfamily\menlo{Menlo}
	\usepackage[scaled]{beramono}
\usepackage[colorlinks, linkcolor=red, anchorcolor=blue, citecolor=red]{hyperref} % hyperlink
\usepackage{fancyhdr} % header and footer
    \pagestyle{fancy}
    \lhead{\newtitle}
    \chead{}
    \rhead{}
    \lfoot{}
    \cfoot{}
    \rfoot{\thepage}
    \renewcommand{\headrulewidth}{0.4pt}
    \renewcommand{\footrulewidth}{0.4pt}
\usepackage{enumitem} % enumerate package
\usepackage[framemethod=TikZ]{mdframed} % frame text
\usepackage{amsthm}
    % R Language Output
    \newcounter{Rop}[section]\setcounter{Rop}{0}
    \renewcommand{\theRop}{\arabic{Rop}}
    \newenvironment{Rop}[2][]{%
    \refstepcounter{Rop}%
    \ifstrempty{#1}%
    {\mdfsetup{%
    frametitle={%
    \tikz[baseline=(current bounding box.east),outer sep=0pt]
    \node[anchor=east,rectangle,fill=blue!20]
    {\strut R~Output~\theRop};}}
    }%
    {\mdfsetup{%
    frametitle={%
    \tikz[baseline=(current bounding box.east),outer sep=0pt]
    \node[anchor=east,rectangle,fill=blue!20]
    {\strut R~Output~\theRop:~#1};}}%
    }%
    \mdfsetup{innertopmargin=10pt,linecolor=blue!20,%
    linewidth=2pt,topline=true,%
    frametitleaboveskip=\dimexpr-\ht\strutbox\relax
    }
    \begin{mdframed}[font=\menlo]\relax%
    \label{#2}}{\end{mdframed}}
% \renewcommand{\thefigure}{\thesection.\arabic{n}} % figure number
\numberwithin{figure}{section} %figure number
\pdfstringdefDisableCommands{% necessary when there are bm and hyperref
    \renewcommand*{\bm}[1]{#1}%
}

%\setlength{\parindent}{0pt} %indent
%%%%%%%%%%%%%%%%%%%%%%%%%%%%%%%%%%%%%%%%%%%%%%%%%%%%%%%%%%%%
\begin{document}
\maketitle
%%%%%%%%%%%%%%%%%%%%%%%%%%%%
\section{Question 3}
Let 
    \[
        A = 
        \begin{bmatrix}
            2 & -1 & 1 \\
            -2 & 1 & 3 \\
            4 & 0 & -1 \\
        \end{bmatrix}
    \]

\subsection{Subquestion i} 
Show that the 2 x 2 leading principal minor of A is 0, i.e., show that
    \[
        det \begin{bmatrix}
            2 & -1 \\
            -2 & -1 \\
        \end{bmatrix}  = 0
    \]

\textbf{Solution} : The determinant of the second leading principal minor is:
    \[
        det \begin{bmatrix}
            2 & -1 \\
            -2 & -1 \\
        \end{bmatrix}  = 2(1) - (-2)(-1) = 0
    \]

\subsection{Subquestion ii} 
Attempt to do the LU decomposition and show division by U(2,2) cannot be done.

\textbf{Solution}: We apply the LU decomposition and find 
    \[
        U(1,1) = A(1,1) = 2 \\
        U(1,2) = A(1,2) = -1 \\
        U(1,3) = A(1,3) = 1 \\
    \]
We calculate the first column of L as:
    \[
        L(k,1) = \frac{A(k,1)}{U(1,1)}, \forall k = 1:n 
    \]

Then,
    \begin{align*}
        L(1,1) &= \frac{2}{2} = 1 \\
        L(2,1) &= \frac{-2}{2} = -1 \\
        L(3,1) &= \frac{4}{2} = 0.5 \\
    \end{align*}
The current LU decomposition is:
    \[
        L = 
        \begin{bmatrix}
            1 & 0 & 0 \\
            -1 & 1 & 0 \\
            2 & L(2,2) & 1 \\
        \end{bmatrix} ,
        U = 
            \begin{bmatrix}
                2 & -1 & 1 \\
                0 & U(2,2) & U(2,3) \\
                0 & 0 & U(3,3) \\
            \end{bmatrix}
    \]
We then update the matrix A by taking out the first column and row of A, L and U. We are left with:
    \[
    \begin{aligned}
        UpdatedA 
        &= 
            \begin{bmatrix}
                1 & 3 \\
                0 & -1 \\
            \end{bmatrix} - 
            \begin{bmatrix}
                -1  \\
                2  \\
            \end{bmatrix}
            \begin{bmatrix}
                -1 & 1 \\
            \end{bmatrix} \\ 
        &= 
            \begin{bmatrix}
                1 & 3 \\
                0 & -1 \\
            \end{bmatrix} - 
            \begin{bmatrix}
                1 & 1 \\
                -2 & 2 \\
            \end{bmatrix} \\
        &= 
            \begin{bmatrix}
                0 & 2 \\
                2 & -3 \\
            \end{bmatrix} 
    \end{aligned}
    \]

We then write:
    \[
        \begin{bmatrix}
            1 & 0 \\
            L(3,2) & 1 \\
        \end{bmatrix}
        \begin{bmatrix}
            U(2,2) & U(2,3) \\
            0 & U(3,3) \\
        \end{bmatrix} =
        \begin{bmatrix}
            0 & 2 \\
            2 & -3 \\
        \end{bmatrix}
    \]
Now,
    \[
        U(2,2) = 0, L(3,2) = \frac{A(3,2)}{U(2,2)}
    \]
which causes the LU decomposition to breakdown with division by zero.

%%%%%%Part iii
\subsection{Subsection iii}
Show that the matrix A is nonsingular.

\textbf{Solution}: We find the determinant of A as -16 $\neq$ 0.

To find the LU decomposition with row pivoting, we look for matrices P, A, L and U such that:
    \[
        PA = LU
    \]
with
    \[
        A = 
        \begin{bmatrix}
            2 & -1 & 1 \\
            -2 & 1 & 3 \\
            4 & 0 & -1 \\
        \end{bmatrix},
        L=
        \begin{bmatrix}
            1 & 0 & 0 \\
            L(2,1) & 1 & 0 \\
            L(3,1) & L(3,2) & 1 \\
        \end{bmatrix},
        U =
        \begin{bmatrix}
            U(1,1) & U(1,2) & U(1,3) \\
            0 & U(2,2) & U(2,3) \\
            0 & 0 & U(3,2) \\
        \end{bmatrix},
        P = I
    \]

We find the largest entry in the first column of A at row 3. We swap rows 1 and 3 of A and P to have:
    \[
        A =
        \begin{bmatrix}
            4 & 0 & -1 \\
            -2 & 1 & 3 \\
            2 & -1 & 1 \\
        \end{bmatrix},
        P = 
            \begin{bmatrix}
                3 \\
                2 \\
                1 \\
            \end{bmatrix}
    \]
We then compute
    \[
        U(1,1) = A(1,1) = 4, U(1,2) = A(1,2) =0, U(1,3) = A(1,3) = 1
    \]
Then,
    \[
        L(2,1) = \frac{A(2,1)}{U(1,1)} = \frac{-2}{4} = -0.5 \\
        L(3,1) = \frac{2}{4)} = 0.5
    \]
Current state
    \[
        L=
            \begin{bmatrix}
                1 & 0 & 0 \\
                -0.5 & 1 & 0 \\
                0.5 & L(3,2) & 1 \\
            \end{bmatrix},
        U =
            \begin{bmatrix}
                4 & 0 & 1 \\
                0 & U(2,2) & U(2,3) \\
                0 & 0 & U(3,2) \\
            \end{bmatrix}
    \]
We now update the matrix A by removing the first row and column of A, L and U:
    \[
        \begin{aligned}
            UpdatedA &= 
                \begin{bmatrix}
                    1 & 3 \\
                    -1 & 1 \\
                \end{bmatrix} -
                \begin{bmatrix}
                    -0.5 \\
                    0.5 \\
                \end{bmatrix}
                \begin{bmatrix}
                    0 & 1 \\
                \end{bmatrix}
            &=
                \begin{bmatrix}
                    1 & 3 \\
                    -1 & 1 \\
                \end{bmatrix} -
                \begin{bmatrix}
                    0 & -0.5 \\
                    0 & 0.5 \\
                \end{bmatrix} 
            &=
                \begin{bmatrix}
                    1 & 3.5 \\
                    1 & 0.5 \\
                \end{bmatrix} 
        \end{aligned}
    \]
Since the largest entry in the first column is in the first row, no pivoting is required. We calculate
    \begin{align*}
        U(2,2) &= UpdatedA(1,1) = 1 \\
        U(2,3) &= UpdatedA(1,2) = 3.5 \\
        L(3,2) &= \frac{-1}{1} = -1 \\
    \end{align*}
Then,
    \[
        L = 
        \begin{bmatrix}
            1 & 0 \\
            -1 & 1 \\
        \end{bmatrix} ,
        U =
        \begin{bmatrix}
            1 & 3.5 \\
            0 & U(3,3) \\
        \end{bmatrix} 
    \]
We take out 1 row and 1 column of UpdatedA, L, and U to have
    \[
        0.5 - (-1)(3.5) = 4.
    \]
We find U(3,3) = 4 and the LU decomposition is:
    \[
        P =
            \begin{bmatrix}
                0 & 0 & 1 \\
                1 & 0 & 0 \\
                0 & 1 & 0 \\
            \end{bmatrix},
        L =
        \begin{bmatrix}
            1 & 0 & 0 \\
            -0.5 & 1 & 0 \\
            0.5 & -1 & 1 \\
        \end{bmatrix},
        U =
        \begin{bmatrix}
            4 & 0 & 1 \\
            0 & 1 & 3.5 \\
            0 & 0 & 4 \\
        \end{bmatrix}
    \]

\end{document}
